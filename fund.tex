\documentclass[UTF8,12pt]{ctexbook}
\usepackage{amsmath,amssymb,geometry,bm,graphicx,fontspec,amssymb,amsthm}
\usepackage[mathscr]{euscript}

\usepackage{tabularray}

\usepackage[colorlinks,
linkcolor=black,
anchorcolor=blue,
citecolor=green
]{hyperref} % 目录中的超链接

\newtheorem{Def}{定义}[section]
\newtheorem{Theo}{定理}[section]
\newtheorem{Lemm}{引理}[section]
\newtheorem{Prop}{命题}[section]
\newtheorem{Assu}{假设}[section]
\newtheorem{Axiom}{Axiom}

\numberwithin{equation}{section} % 按章节进行排序与编号
\numberwithin{figure}{section}
\numberwithin{table}{section}

\usepackage{draftwatermark} % 所有页加水印
\SetWatermarkText{Fanchile} % 设置水印内容
\SetWatermarkLightness{0.99} % 设置水印透明度 0-1
\SetWatermarkScale{1} % 设置水印大小    

\title{基金从业} % 文档相关信息
\author{EconNerd}
\date{\today}
\geometry{scale=0.8}

\begin{document}
	\maketitle
	\tableofcontents
	\newpage
	\chapter{证券投资基金}
	\section{序言}
	
	\chapter{股权投资基金}
	\section{股权投资基金概述}
	
	\subsubsection{基金是干什么的}
	
	
	股权投资基金就是投资于私人股权(Private Equity)的投资基金,即未公开发行和交易的股权。
	
	基金的募集方式也可以分为非公开方式募集和公开方式募集。但是在我国目前只能以非公开方式募集。
	
	对于投资人来说,通常股权投资基金投资期限长、流动性差、收益波动率高。
	
	
	\subsubsection{股权投资的发展历史}
	
	股权投资起源于创业投资,1946年成立的美国研究与发展公司(ARD)被公认是全球第一家以公司形式运作的创业投资基金。
	
	1958年美国《小企业投资法》(Small Business Investment Act)的颁布是股权投资基金发展的里程碑。根据该法,联邦政府为小企业申请贷款提供了宽松的政策支持,联邦政府下设小企业管理局(Small Business Administration)其中办理小企业贷款申请业务。同年,该管理局设立小企业投资公司计划(SBIC),以低息贷款和融资担保的形式鼓励成立小企业投资公司。
	
	1973年美国创业投资协会成立,标志着创业投资在美国发展称为专门行业。
	
	1976年KKR成立以后,开始出现专门从事并购投资的并购投资基金。20世纪80年代美国第四次并购浪潮催生了黑石(1985)、凯雷(1987)和德太投资(1992)等著名并购基金管理机构。
	
	2007年,数家并购基金脱离美国创业投资协会,发起设立了主要服务于并购基金管理机构的美国私人股权投资协会(PEC)
	
	我国股权投资基金发展历史可以分为三个阶段
	
	1985-2004 探索与起步阶段
	
	
	2005-2012 快速发展阶段
	
	2009年创业板。2007年合伙企业法
	
	2013年至今 通义监管下的规范化发展阶段
	
	2013年6月,中央编办发出《关于私募股权基金管理职责分工的通知》,明确由证监会统一形式股权投资基金监管职责。
	
	2014年8月,中国证监会发布《私募投资基金监督管理暂行办法》
	
	
	
	\subsubsection{基本运作的模式}
	模式就是四个字 募投管退
	
	\subsubsection{从宏观上的作用是什么}
	第一个作用就是解决了中小企业融资的问题,而解决了融资问题变相帮助了创新企业和创新行业的发展
	
	第二个就是在并购方面对于产业转型和升级的贡献
	
	\section{股权投资基金参与主体}
	参与的主体分为三类,一类是和基金业务直接相关的当事人。另外的两类则分别是中介服务机构以及监管机构
	
	\subsection{当事人}
	主要是投资者、管理人以及托管人
	
	主要关注一下托管人。为了保证基金财产的安全,有些国家的法律法规要求股权投资基金按照资产管理和保管分开的原则进行运作,由专门的基金托管人保管基金资产。
	
	
	\subsection{市场服务机构}
	服务机构主要包括基金财产保管机构、基金销售机构、基金份额登记机构、律师事务所、会计事务所。
	
	\subsection{监管机构和自律组织}
	
	监管的目标主要是保护基金投资者合法权益以及防范系统性金融风险
	
	\section{股权投资基金分类}
	根据不同分类标准可以把股权投资基金进行不同的分类。
	
	\subsection{创业投资基金和并购基金}
	从投资标的来分可以分为创业投资基金和并购基金。
	
	创业投资基金主要投资于未上市的成长性企业,不借助杠杆进行参股投资。收益来源则主要是股权增值
	
	而并购基金则相反。并购基金主要是杠杆收购(LBO)基金。主要投资于成熟企业,采用控股投资。且往往借助于杠杆
	
	杠杆收购的资金来源主要有三种。1.普通股 2.夹层资本可包括优先股和垃圾债券3. 高级债,即银行提供的并购贷款
	
	\subsection{公司型、合伙型和信托型}
	基金也可以采用不同的组织形式
	
	主要讲一下信托型。这种基金是指通过订立信托七月的形式设立的股权投资基金。基金投资者通过购买基金份额,享有基金投资收益。基金管理人依据基金合同负责基金的经营和管理操作。
	
	信托主要有以下特点
	1. 基金本质是一种信托关系,比较灵活
	2. 信托基金不是纳税主题,只有投资者需要对投资收益缴纳所得税,避免了双重纳税
	
	\subsection{人民币基金和外币基金}
	
	指的注意的是外币基金通常采取两头在外的方式。即基金经营实体注册在境外,被投资对象在境外设立特殊目的公司作为受资对象,并且在境外完成退出
	
	\subsection{股权投资母基金和政府引导基金}
	母基金就是以股权投资基金为主要投资对象的基金
	
	20世纪70年代就出现了单个投资者签订委托投资协议,将资金投向多个股权投资基金的情况
	
	20世纪90年底初,真正意义上的股权投资母基金开始发展起来。这个可以类比二级市场中的FOF
	
	母基金的业务主要包括一级投资、二级投资和直接投资。
	一级投资是在基金募集时投资、二级投资是对存续基金进行投资。直接投资就是直接对非公开发行和交易的企业股权进行投资
	
	
	母基金有以下几个作用:1.分散风险、2.专业管理、3.投资机会、4.规模优势
	
	政府引导基金是一类特殊的母基金。对基金的支持方式也包括残酷、融资担保、跟进投资
	
	\section{股权投资基金的募集与设立}
	
	\section{股权投资基金的投资}
	
	\section{股权投资基金的投后管理}
	
	\section{股权投资基金的项目退出}
	
	\section{股权投资基金的内部管理}
	
	\section{股权投资基金的政府管理}
	
	\section{股权投资基金的行业自律}
	
	
	
	
\end{document}